%\documentclass[referee]{aa} % for a referee version
%\documentclass[onecolumn]{aa} % for a paper on 1 column  
%\documentclass[longauth]{aa} % for the long lists of affiliations 
%\documentclass[letter]{aa} % for the letters 
\documentclass{aa}

\usepackage{txfonts}
\usepackage{natbib}

\usepackage{graphicx}

\usepackage{color}
\usepackage{hyperref}
\hypersetup{colorlinks=true,allcolors=[rgb]{0,0,0.8}}

% added to hide columns in the tables
\usepackage{array}
\newcolumntype{H}{>{\setbox0=\hbox\bgroup}c<{\egroup}@{}}

\usepackage{appendix}

% align columns in table
\usepackage{makecell}

\usepackage{showyourwork}

% the three lines suppress the hyperref 'link empty' warnings
% explanation at: https://tex.stackexchange.com/questions/345764/journal-class-shows-package-hyperref-warning-suppressing-link-with-empty-targe
%\makeatletter
%\renewcommand*\aa@pageof{, page \thepage{} of \pageref*{LastPage}}
%\makeatother

% text highlighting
%\usepackage{soul}
%\sethlcolor{yellow}


\newcommand{\asas}{ASASSN-21qj}
\newcommand{\ktwo}{\textit{K2}}
\newcommand{\kms}{km~s$^{-1}$\xspace}
\newcommand{\ms}{m~s$^{-1}$}
\newcommand{\gcc}{g~cm$^{-3}$}
\newcommand{\masyr}{mas~yr$^{-1}$}
\newcommand{\err}{\textit{$\pm$}}
\newcommand{\teff}{$T_\mathrm{eff}$}
\newcommand{\msun}{$M_\odot$}
\newcommand{\rsun}{$R_\odot$}
\newcommand{\lsun}{$L_\odot$}
\newcommand{\rhosun}{$\rho_\odot$}
\newcommand{\mstar}{$M_*$}
\newcommand{\rstar}{$R_*$}
\newcommand{\lstar}{$L_*$}
\newcommand{\rearth}{$R_\oplus$}
\newcommand{\vrad}{$v_{R}$}
\newcommand{\pmra}{$\mu_{\alpha}$}
\newcommand{\pmdec}{$\mu_{\delta}$}

\newcommand{\rhostar}{$\rho_*$}
\newcommand{\mjup}{$M_\mathrm{Jup}$}
\newcommand{\galex}{\textit{GALEX}}
\newcommand{\gaia}{\textit{Gaia}}
\newcommand{\kepler}{\textit{Kepler}}
\newcommand{\spitzer}{\textit{Spitzer}}
\newcommand{\ktwosc}{\textsc{k2sc}}
\newcommand{\ktwosff}{\textsc{k2sff}}
\newcommand{\hipparcos}{\textit{Hipparcos}}
\newcommand{\tess}{\textit{TESS}}
\newcommand{\emcee}{\textsc{emcee}}
\newcommand{\python}{\textsc{python}}


\begin{document} 
\authorrunning{Hoogenboom et al.}
\titlerunning{Brightest southern star light curves}

   \title{Caveat Emptor: light curves of the brightest stars in the \\ Southern sky as seen by bRing}

   \author{Rosa Hoogenboom
          \inst{1}
          \and
          Sanna Heesakkers
          \inst{1}
          \and
          Matthew A. Kenworthy
          \inst{1}
          \and
          Remko Stuik
          \inst{1}
          \and
          Ignas Snellen
          \inst{1}
          \and
          Patrick Dorval
          \inst{1}
          }

          \institute{Leiden Observatory, Leiden University, PO Box 9513, 2300 RA Leiden, The Netherlands\\
             \email{kenworthy@strw.leidenuniv.nl}
              }

   \date{Received \today; accepted XXXX}

% \abstract{}{}{}{}{} 
% 5 {} token are mandatory
 
  \abstract
  % context heading (optional)
  % {} leave it empty if necessary  
   {The brightest stars in the sky are also amongst the most difficult to obtain long-term precision photometry. }
  % aims heading (mandatory)
   {We provide high cadence light curves with an individual photometric precision of 0.5\% for 157 stars in the Southern hemisphere from the bRing robotic observatories.}
  % methods heading (mandatory)
   {We process the bRing images and remove the systematic trends due to the large pixel sizes, the varying night sky background, cloud coverage and intrapixel variations, in addition to diurnal, synodic and seasonal variations.}
  % results heading (mandatory)
   {We present 154 light curves for stars $m_V<4$ for all stars below $\delta \leq -30\degr $ covering baselines up to 3 years. The data is obtained by the $\beta$ Pictoris b Ring project (bRing), after which it is reduced and detrended.}
  % conclusions heading (optional), leave it empty if necessary 
   {}

   \keywords{Instrumentation: photometers --- Techniques: photometric}

   \maketitle
%
%-------------------------------------------------------------------

\section{Introduction}

Long term photometric observations of the brightest stars in the sky is challenging: the large photon rate on even the smallest telescopes can saturate most modern electronic detectors that are optimised for flux rates typically a million times lower.
%
A second problem is the lack of calibration stars of similar brightness on the celestial sphere that enable photometric calibration with a typical surface density of one star per 25 square degrees for the naked eye stars.
%
All sky cameras used for tracking meteors can provide unsaturated photometry of the brightest stars but their small entrance pupil means that they are susceptible to scintillation noise, limiting the photometric precision.

The Beta Pictoris Ring \citep[bRing; ][]{Stuik_2014} project is a network of four cameras in two completely automated observatories designed to provide continuous calibrated photometry of the exoplanet host star Beta Pictoris.
%
The Hill sphere of the gas giant Beta Pictoris b transited the star in 2015--2016, and the bRing observations provided significant upper limits on the presence of circumplanetary material around the planet \citep{Kenworthy21}.
%
Unsaturated light curves are available up to a magnitude of 4, but the bRing cameras saturate for brighter stars near the zenith.
%
At higher airmasses the atmospheric extinction allows the brightest stars to become unsaturated, and combined with the all-sky photometric calibration pipeline used for the MASCARA and bRing telescopes, the unsaturated data can be extracted to form light curves covering the duration of the bRing project, currently 2015 to 2022.

This paper presents long term photometric light curves for all naked sky stars visible from the Southern hemisphere.
%
In Section \ref{sec:obs} we summarise the raw data and initial data pipeline.
%
We then discuss the filters we use to remove saturated photometry and to correct for the known systematics in the bRing observatory.
%
In Section \ref{sec:methods}

% In 2014 the Kepler K2 mission was launched to detect transits on stars with a magnitude brighter than 12 \citep{Howell_2014}.
% %
% Stars with $m_V<8$ could not be observed until the creation of the Multi-site All-Sky CAmeRA (MASCARA) \citep{mascara}.
% %
% When the MASCARA project was created in 2015 to detect exoplanets around stars with $8 > m_V > 4$ \citep{Talens_2017}, much brighter stars could be observed.
% %
% MASCARA uses long exposure times of 6.38 seconds to observe the sky which results in saturation of the pixels for stars with $m_V < 4$.
% %
% This changed in 2017 when the $\beta$ Pictoris b Ring project (bRing) \citep{bRing} was introduced.
% %
% The MASCARA and bRing projects both use exposure times of 6.38 s.
% %
% However, bRing also observes stars using an exposure time of 2.54 s.
% %
% Because of these short exposure times, bRing is able to collect data of stars with a magnitude brighter than 4 in the southern hemisphere.


% The bRing project can observe this star, because bRing was created to look at $\beta$ Pictoris specifically.
% %
% This small amount of discovered exoplanets suggests that there are still opportunities left to find new exoplanets around bright stars observed by bRing.

% The aim of this research is to provide a catalogue of long-term light curves of the brightest stars in the southern hemisphere suitable for research into stellar pulsations, variable stars and eclipsing systems.


%
% We use the data taken by the bRing cameras, for stars with a magnitude brighter than 4.
% %
% After creating light curves for the bright stars, we use the Transit Least Squares algorithm \citep{Hippke_2019} and Lomb-Scargle periodograms \citep{VanderPlas_2012,VanderPlas_2015} to try to find periodic signals in the data.
% %

%In \autoref{sec:methods} we explain the bRing project, the steps we take in our data reduction and how the data is detrended.
%
%In \autoref{sec:tlslomb} the Transit Least Squares algorithm and Lomb-Scargle periodograms are explained.
%
%We present our results in \autoref{sec:results}.
%
%We discuss our findings in \ref{sec:discussion} and give a conclusion in \ref{sec:conclusion}.
%
%We describe our collaboration during the research project in \ref{sec:collaboration}.
%
%A table containing information about the stars with a magnitude brighter than 4 is given in \ref{appendix:table}.
%
%The light curves of all these bright stars are given in \ref{appendix:lightcurves}.

\section{Observations}\label{sec:obs}

\subsection{bRing}

The bRing project is based on the MASCARA project \citep{mascara} which takes images of the sky above airmass 2.
%
However, while MASCARA uses five CCDs (facing north, south, east, west and central) to take images of the whole visible sky in the northern hemisphere, bRing uses only two (facing east and west).
%
The bRing telescopes are situated in Australia and South Africa, thus positioned to look at the southern hemisphere ($\delta \leq -30 \degr $).
%
The primary goal of the two projects also differs: MASCARA was created to search for transiting exoplanets around bright stars ($4 < m_V < 8$) and bRing was constructed to monitor the Hill sphere transit of exoplanet $\beta$ Pictoris b in front of the star $\beta$ Pictoris.
%
The transit took place from April 2017 lasting 300 days until March 2018 \citep{Wang_2016}.
%
Because of this period bRing started collecting data on 17 January 2017 \citep{bRing} and has continued every clear night since then.

While observing $\beta$ Pictoris, bRing observes 20000 additional bright stars and uses a custom pipeline to provide photometric calibration for the visible sky that accounts for several systematic effects.
%
The bRing project also monitors the atmospheric transmission and clouds.
%
To capture both bright and relatively fainter stars, bRing uses two exposures times: 6.38 s for relatively faint stars ($4 < m_V < 8$) and 2.54 s for the brightest stars ($m_V < 4$). This results in a total of 13500 data points per day.
%
These data points are calibrated every night by the bRing Control Software and sent to Leiden, the Netherlands.

\begin{figure}
    \centering
    \includegraphics[width=0.90\columnwidth]{figures/bRing.jpg}
    \caption{The bRing instrument. Reprinted from \cite{bRing}, figure 5.}
    \label{bRing}
\end{figure}

\subsection{The stars}

All stars were selected that have $m_V<4$ and have a declination < 30 degrees, resulting in 157 stars, marked in red in the Hertzsprung-Russell (HR) diagram in Figure~\ref{HRD}.
%
The majority of the stars with an apparent magnitude brighter than 4 are giants, found in the upper part of the HR diagram.
%
Only a few of them are main sequence or late type dwarfs.


\subsection{Data processing}

TODO get the script for the HR diagram and make sure it still runs.

The data processing from the observatory to the local computers are detailed in \citet{Stuik_2014}.
%
The raw photometric data for each star is stored in a custom database which can be queried and the data retrieved with Python 2.7 scripts.
%
Each star is observed with multiple cameras distributed across the two bRing observatories.
%
We extract the raw data for each camera and save as a FITS file with the format {\tt <HD number>\_<cameraID>.fits} where {\tt HD number} is the Henry Draper number for the star and {\tt cameraID} is one of the four camera names {\tt AUE,AUW,SAE,SAW}.
%
These raw data are written out using the {\tt save\_raw\_data\_from\_bRing\_database.py} script.
%
The resulting 628 data files are 144 Gb in size and are retrievable from the public repository Zenodo at XXXXXX.


% %
% A little over half of the bright stars lie on the main sequence and on the right of the main sequence a lot of giants can be found.
% %
% For reference, the Sun and Vega are also displayed in the figure.
% %
% Vega is one of the brightest stars in the northern hemisphere with a B-V colour of 0.
% %
% Using the B-V colour, the bright stars are found to have spectral type B to K.
% % %
% A few stars have spectral type M. 
\begin{figure}
    \centering
    \includegraphics[width=0.90\columnwidth]{figures/hrd.pdf}
    \caption{This plot presents 154 of the 157 stars for which we make light curves in red.
      %
      The grey dots represent all of the stars which are observed by bRing.
      %
      The Sun is plotted in yellow and Vega in light blue (both of which are not observed by bRing).
      %
      On the vertical axis the absolute magnitude is displayed and on the horizontal axis the B-V colour scale is shown.}
    \label{HRD}
    \script{plot_HRD.py}
  \end{figure}

\subsection{Rejection of points}

%To make long-term light curves, we started with data reduction, which we wrote using Python 2.7.
%
The data reduction consists of multiple steps.
%
For each step, we used the data of one of the 4 bRing cameras at a time.
%
The primary calibration from \cite{Talens_2018} was applied.
%
For each star, there are two data sets generated by the calibration called ``daily'' data and ``quarterly'' data.
%
The daily data consists of all data points collected by the bRing cameras, collected with a cadence of 6.35 seconds, enabling searches for short term photometric fluctuations.
%
The quarterly data bins the photometry to every 50 data points, enabling searches for long term trends.
%
We use the daily data since the bright stars have very little quarterly data due to the binning including saturated photometric points.

Variability is seen in the daily data due to several systematic effects.

%
This results in a distribution of data points around the mean value. To determine whether a data point should be rejected, the standard deviation was used.
%
When removing data points based on standard deviation, we used the Astropy function \texttt{sigma\_clip}.
%
The mean of the data set was calculated iteratively, discarding outliers each time.
%
Based on the final calculated mean, outliers were removed from the data set.
%
We also used this sigma-clipped-mean in further calculations when needed, instead of the mean which can be calculated before sigma-clipping.

We started the data reduction by selecting good data points from the daily and quarterly data using these constraints:

\begin{enumerate}
    \item All data points from the eastern Australian camera on 3 and 4 April 2018 and the eastern South African camera on 10 September 2020 were discarded due to significant outliers of unknown origin.
    \item The 2.35 second exposures were used.
    %\item The data points where the cloud error was 3$\sigma$ above the mean cloud error were deleted.
    %
    %The 3$\sigma$ cutoff was estimated by eye.
    %
    %The cloud error tells us how the transmission changes over time; a big change might mean a cloud is in view.
    %
    %These outliers usually indicate that the clouds are variable at that point in time, in which case the data should not be used.
    \item The 3$\sigma$ outliers of the clouds were removed. %This results in less background noise.
    \item The 5$\sigma$ outliers of the sky background were rejected. %This rejection also results in less background noise.
    \item The 5$\sigma$ outliers based on magnitude error were discarded. %A big error could mean that the measurement is wrong.     
    \item All data where the Sun is lower than $-18\degr$ was selected. %This was added to ensure that the sky background did not get too bright.     
    \item We made sure that the magnitude error for the daily data is always greater than zero and finite, so no incorrect values are encountered. 
    \item We used the astrometry and photometry flags of the daily data. 
    %
    These tell us if the measured flux is too high or too low to be realistic or if no photometry was performed due to, e.g., the star being too close to the edge of the image.
    %
    Only the data without any bad flags was used. 
\end{enumerate}

After selecting the data points that pass all these steps in both the daily and quarterly data, the daily data was calibrated.
%
We cannot use the daily data to remove, for example, the clouds, because those values are only available in the quarterly data, so the missing information was copied from the quarterly data to the daily data. 

The impact of each step of the reduction is shown in Figure~\ref{fig:reduction}.

\begin{figure*}
    \centering
    \includegraphics[width=\linewidth]{figures/reduction.pdf}
    \caption{The first three steps of the data reduction are shown here. The first plot shows the raw data that was saved by bRing. The second plot shows the data without the long exposures. Lastly, the data without the bad photometry and astrometry flags is plotted. The time is displayed on the horizontal axis in Julian Date and the apparent magnitude is displayed on the vertical axis.}
    \label{fig:reduction}
    \script{plot_reduction.py}
\end{figure*}

\subsection{Detrending}

After performing data reduction, the data needed to be detrended due to effects not caused by the star itself.
%
For daily trends the effects are caused by the varying shape of the point spread function and its illuimination of the lenslets on the bRing detector cameras, which changes the amount of light within the photometric aperture.
%
For monthly trends the effects are caused by the scattered light when the Moon is visible at the site.

Three trends were removed: two daily trends, and the monthly trend caused by the Moon.
%
We performed detrending similar to the secondary calibration used in \cite{Talens_2018}.

Typically 13500 exposures are taken in one night.
%
Every exposure at the same local sidereal time is given a number, referred to as the {\tt LSTSEQ} \citep{Talens_2018}.
%
During each night, the star should make the same path on the CCD and cross the same pixel at the same local hour angle.
%
Detrending removes the effects caused by the same pixel receiving less or more light consistently.
%
For every unique LSTSEQ, a sigma-clipped mean can be found.
%
Subtracting this mean at every LSTSEQ removes this particular trend.
%
We detrended this trend per year, because the transmission changes slightly over time due to envrionmental effects of the observatory warming and cooling leading to small misalignments, and to the accumulation of dust on the entrance windows.


The detrending of the second daily trend removes the effects caused by the Sun rising and setting.
%
We divided a day into 500 bins, and determined the sigma-clipped mean of the data points in each bin.
%
Five iterations of a sigma clipping altorithm with a threshold of 5 sigma are applied.
%
%Each time sigma-clipping is used for detrending, the default values of 3$\sigma$,  and 5 iterations are applied.
%
The mean was then subtracted from all data points in one bin.

The last trend we removed is the one caused by the moon, which changes the sky background.
%
Trends matched to the synodic period (29.5306 days) were fitted and removed.
%
At this stage, the light curve for each star has all known trends removed.
%
For variable stars with a known period, we removed the large amplitude periodic variations before carrying out the detrending listed above, and then added this offset back into the light curve after the reduction steps.
%
Examples of the detrending steps are shown in the figures below.

%XXXX wait for detrendsteps_plot.py

\begin{figure*}
    \centering
    \includegraphics[width=\linewidth]{figures/84810.pdf}
    \caption{Example of a variable star seen in all the cameras.
    %
    Top panel is all combined light curves, lower panels are the individual cameras.}
    \label{fig:cleaned_output}
    \script{plot_lightcurves.py}
\end{figure*}

\section{Results}
\label{sec:results}

\subsection{Representative light curves}
The light curves of the several brightest stars on the southern hemisphere are presented.

XXX in a Rosa email what are the numbers

message://%3c1131158490.6158.1695733941581@mail.ziggo.nl%3e

I believe the HR-diagram was indeed made with Gaia data.

In order to make the plots in the ``Representative light curves'' section, we could use the makelightcurve.py script that we shared before. You will only have to alter the number all the way at the bottom of the script in "plotLightCurve", which has to be the ASCC number of the star that you want to plot.

These are the stars and their ASCC numbers that you requested:

The brightest star we have: Canopus: ASCC 2111805
Delta Pavonis: ASCC 2424942
Two interesting variable stars: zeta Phoenicis, eclipsing binary: ASCC 2198437
and l Carinae, cepheid: ASCC 2311128

Attached are the code for the reduction-steps plot and the HR-diagram.

The brightest star you have.

Delta Pavonis.

Two interesting variable stars.

%\begin{enumerate}
% \item \ref{Canopus} shows the light curve of alpha Carinae, the second brightest star in the sky and the brightest star in the southern hemisphere. It is better known as Canopus. The light curve can be seen in.
% \item \ref{Rigil_Kentaurus} shows the light curve of alpha Centauri A, the third brightest stars in the sky. It is also known as Rigil Kentaurus. It is part of the alpha Centauri system, which contains Proxima Centauri, the closest star known to the Sun. 
% \item \ref{Achernar} presents the light curve of alpha Eridani, also known as Achernar.
% \item \ref{Hadar} shows the light curve of beta Centauri, also called Hadar. This star is a Beta Cephei.
% \item \ref{Acrux} presents the light curve of alpha Crucis, a binary star system also known as Acrux.
% \item \ref{Mimosa} shows the light curve of beta Crucis, also known as Mimosa.
% \item \ref{Shaula} shows the light curve of lambda Scorpii, also known as Shaula. This is a binary system, with a Beta Cephei as primary star.
% \item \ref{Gacrux} displays the light curve of gamma Crucis, also called Gacrux.
% \item \ref{Miaplacidus} shows the ninth brightest star, beta Carinae. It is also known as Miaplacidus.
% \item \ref{Alnair} shows the final brightest star. This is alpha Gruis, also known as Alnair.

% \begin{figure}
%     \centering
%     \includegraphics[width=\textwidth]{figures/Canopus_thesis.png}
%     \caption{Light curve of Canopus after data reduction and detrending. The time is displayed on the horizontal axis in Julian Date and the apparent magnitude is displayed on the vertical axis. The ASCC-number of Canopus is 2111805.}
%     \label{Canopus}
% \end{figure}

% \begin{figure}
%     \centering
%     \includegraphics[width=\textwidth]{figures/Rigil_Kentaurus_thesis.png}
%     \caption{Light curve of Rigil Kentaurus after data reduction and detrending. The time is displayed on the horizontal axis in Julian Date and the apparent magnitude is displayed on the vertical axis. The ASCC-number of Rigil Kentaurus is 2348878.}
%     \label{Rigil_Kentaurus}
% \end{figure}

% \begin{figure}
%     \centering
%     \includegraphics[width=\textwidth]{figures/Achernar_thesis.png}
%     \caption{Light curve of Achernar after data reduction and detrending. The time is displayed on the horizontal axis in Julian Date and the apparent magnitude is displayed on the vertical axis. The ASCC-number of Achernar is 2199019.}
%     \label{Achernar}
% \end{figure}

% \begin{figure}
%     \centering
%     \includegraphics[width=\textwidth]{figures/Hadar_thesis.png}
%     \caption{Light curve of Hadar after data reduction and detrending. The time is displayed on the horizontal axis in Julian Date and the apparent magnitude is displayed on the vertical axis. The ASCC-number of Hadar is 2345078.}
%     \label{Hadar}
% \end{figure}

% \begin{figure}
%     \centering
%     \includegraphics[width=\textwidth]{figures/Acrux_thesis.png}
%     \caption{Light curve of Acrux after data reduction and detrending. The time is displayed on the horizontal axis in Julian Date and the apparent magnitude is displayed on the vertical axis. The ASCC-number of Acrux is 2333718.}
%     \label{Acrux}
% \end{figure}

% \begin{figure}
%     \centering
%     \includegraphics[width=\textwidth]{figures/Mimosa_thesis.png}
%     \caption{Light curve of Mimosa after data reduction and detrending. The time is displayed on the horizontal axis in Julian Date and the apparent magnitude is displayed on the vertical axis. The ASCC-number of Mimosa is 2250231.}
%     \label{Mimosa}
% \end{figure}

% \begin{figure}
%     \centering
%     \includegraphics[width=\textwidth]{figures/Shaula_thesis.png}
%     \caption{Light curve of Shaula after data reduction and detrending. The time is displayed on the horizontal axis in Julian Date and the apparent magnitude is displayed on the vertical axis. The ASCC-number of Shaula is 1880898.}
%     \label{Shaula}
% \end{figure}

% \begin{figure}
%     \centering
%     \includegraphics[width=\textwidth]{figures/Gacrux_thesis.png}
%     \caption{Light curve of Gacrux after data reduction and detrending. The time is displayed on the horizontal axis in Julian Date and the apparent magnitude is displayed on the vertical axis. The ASCC-number of Gacrux is 2248482.}
%     \label{Gacrux}
% \end{figure}

% \begin{figure}
%     \centering
%     \includegraphics[width=\textwidth]{figures/Miaplacidus_thesis.png}
%     \caption{Light curve of Miaplacidus after data reduction and detrending. The time is displayed on the horizontal axis in Julian Date and the apparent magnitude is displayed on the vertical axis. The ASCC-number of Miaplacidus is 2386073.}
%     \label{Miaplacidus}
% \end{figure}

% \begin{figure}
%     \centering
%     \includegraphics[width=\textwidth]{figures/Alnair_thesis.png}
%     \caption{Light curve of Alnair after data reduction and detrending. The time is displayed on the horizontal axis in Julian Date and the apparent magnitude is displayed on the vertical axis. The ASCC-number of Alnair is 2098110.}
%     \label{Alnair}
% \end{figure}
% \end{enumerate}

Two stars are very close to the Sun in the HR diagram in \ref{HRD}.
%
These stars are delta Pavonis and Beta Hydri.
%
Delta Pavonis has a luminosity of 1.22 L$_\odot$ \citep{Bruntt_2010}, a mass of 0.991 M$_\odot$ and a radius of 1.22 R$_\odot$ \citep{2008yCat..21680297T}.
%
It is a subgiant star which will relatively soon start to become a red giant.
%
Beta Hydri has a luminosity of 3.49 L$_\odot$, a mass of 1.08 M$_\odot$ and a of radius 1.81 R$_\odot$ \citep{Brand_o_2011}.
%
Beta Hydri is also a subgiant star. 

% \begin{figure}
%     \centering
%     \includegraphics[width=\linewidth]{figures/del_Pav_thesis.png}
%     \caption{Light curve of delta Pavonis after data reduction and detrending. The time is displayed on the horizontal axis in Julian Date and the apparent magnitude is displayed on the vertical axis. The ASCC-number of delta Pavonis is 2424942.}
%     \label{del_Pav}
% \end{figure}

% \begin{figure}
%     \centering
%     \includegraphics[width=\textwidth]{figures/bet_Hyi_thesis.png}
%     \caption{Light curve of beta Hydri after data reduction and detrending. The time is displayed on the horizontal axis in Julian Date and the apparent magnitude is displayed on the vertical axis. The ASCC-number of beta Hydri is 2430714.}
%     \label{bet_Hyi}
%   \end{figure}

\section{Discussion}

\label{sec:discussion}

157 light curves of the brightest stars ($m_V < 4$) in the southern hemisphere ($\delta \leq -30^o$) were created.
%
This is the first time light curves have been produced for such bright stars.
%
We discuss our results in the subsections below. 
%
Limitations and strengths of the methods used are given as well.

\subsection{Data reduction}

\label{sec:discdatared}
% The data reduction still has a few shortcomings.

% Normally, the last step of the data reduction would be to remove the $5 \sigma$ magnitude outliers. To remove these outliers, the mean of the magnitude should be calculated as accurately as possible. To determine this mean, the best calibrated day in the data for that star should be used. The data points for this day should lie on a horizontal line, without any clear outliers. We defined this day to have a cloud error within 0.5$\sigma$ of the mean cloud error. This constraint was chosen by eye. The day which had the most data points within the limitation should be selected.

% However, a lot of data points seem to be concentrated around a magnitude value which is used to fix the mean magnitude. This value originates from the bRing catalogue \citep{Talens_2018}. The concentration of data points might be caused by a lack of data points, meaning too many will be fixed around this value. In the light curves, this can be seen as a stripe or blob of data points concentrated around a fixed value. This results in an unusually small standard deviation, making good data points seem outliers when rejecting $5\sigma$ outliers.

% For some stars, removing the $5 \sigma$ outliers results in a cleaner light curve. However, for most stars, too many accurate data points are removed in this process, which can be seen in the histogram in \ref{hist}. A sharp cutoff is visible at the higher magnitudes. A peak can be seen at the magnitude value originating from the bRing database as well ($m_V = 3.848$).

% If a solution to this problem is discovered in the future, it should be possible to adapt the reduced data by adding in another data reduction step. The data reduction steps taken in this research could still be used in this case, and do not have to be repeated.
% \begin{figure}
%     \centering
%     \includegraphics[width=\textwidth]{figures/thesis_histogram.png}
%     \caption{Histograms showing the data before and after removing $5\sigma$ magnitude outliers. 200 bins were used for both histograms. The bRing catalogue magnitude value is 3.848 for this star, indicated by the dashed vertical line. The horizontal axis shows the magnitude. The ASCC-number for upsilon$^1$ Centauri is 1959875.}
%     \label{hist}
% \end{figure}
% Another flaw in the data is that the values of the apparent magnitudes may not be accurate compared to other sources. An error remains in the calibration of the magnitudes, and therefore both absolute magnitude values as well as relative changes in magnitude are not completely accurate.
% This is because the mean magnitude of the star is fixed to the visual magnitude in the bRing catalogue \citep{Talens_2018}, which is not always correct. The overall shape of fluctuations in magnitude is relatively good, but to determine the depth of transits and periods, follow-up observations are needed.

% Furthermore, for the eastern Australian camera, the magnitude of some stars reaches values around $10^{19}$ or higher on one day, which are clearly too extreme. This could be caused by some error in the measurements. The scattering is reduced by removing the 100$\sigma$ magnitude outliers. This high value is chosen, because a smaller value would remove some of the good data points. 


Intrapixel variations were neglected in the data reduction.
%
These quasi-sinusoidal modulations are caused by the interline design of the CCDs \citep{Talens_2018}.
%
The intrapixel variations cause noise in the light curves, but are too complex to remove in this research.

% The brightest stars which are visible high in the sky have less data points in general. The data at the beginning and end of the night is sufficient to use in the light curves, but the data in the middle of the night cannot be used. These data points are usually saturated, because the star is at its brightest when it is positioned high in the sky.

% In contrast, less bright stars which have a declination close to the horizon have fewer data points than stars higher in the sky. This is because they are visible for a smaller amount of time each night. The amount of data points observed still differs for each camera, because of the different locations of the cameras. Overall, the South African eastern camera contains the most data points per star. This is why we mostly use this camera to present the light curves.

\subsection{Periodic and variable stars}

After detrending the calibrated data, the mean magnitude of the stars is zero.
%
For variable stars this is prevented by adding the offset of the period back into the detrended data.
%
To fix the mean for the stars which are not variable, the mean should be calculated from the data after data reduction is completed, but before detrending is performed.
%
This mean should then be added to the detrended data.
%
However, calculating the mean from the calibrated data does not return the most accurate magnitude value, because of the values from the bRing database mentioned in \ref{sec:discdatared}.

Stars with no known periods were assumed to be non-variable stars and detrended as such.
%
Possible undiscovered periods can be found by using the data available before detrending.
%
Variable stars with a period longer than 500 days were not detrended for their period.
%
If an eclipse is visible in the light curve, it can only show up for a maximum of one time, which means detrending would not have an effect.
%
Pulsating stars with a period below one day were also detrended as non-variable stars.
%
These short periods of a few hours are not visible in the Lomb-Scargle periodograms.
%
The true period could be dominated by systematic effects within one day which cause stronger signals.
%
This is because after detrending, periodic signals of a fraction of a day still remain.
%
Periodic signals of a multiple of one day still remain as well.
%
These signals should be handled carefully when searching for variability in a light curve.
%
No detrending was conducted for the sidereal period of the moon, which is about 27.3 days.
%
This peak is very rarely observed in periodograms.
%
If this peak is still found, it should be treated especially carefully, because we did not remove this signal in the detrending process.

% \subsection{Searching for periodic signals}
% Periodograms could not be made for all 157 stars, because this process takes a long time. This means that undiscovered periodic signals may still remain in the light curves.

% Variable stars with an irregular period are hard to identify by looking at periodograms. Multiple peaks can be found, but period folding does not show a clear shape in the light curve.

% Not all signals indicate a true period.
% When a period is visible as a signal, a signal will also be visible at multiples of this period. For eclipsing binaries, half of the period may be detected as a signal, due to the overlapping of the primary and secondary period.
% %
% Lomb-Scargle signals around one year should also be treated carefully when examining periodograms. Usually a broad signal is detected around this period, because large gaps of data overlap. These gaps exist simply because the star is not visible on the sky for several months.

% Likewise, a dip in the light curve of a star does not always mean a celestial body moves in front of a star. This drop could be the result of an unclean lens or clouds moving in front of the camera, for example. By looking at data from a different camera or nearby stars using the same camera this theory could be tested. If the dip is not present in the data taken by other cameras, there is no transit occurring. If the light curves of nearby stars show the same drop, the dip is caused by systematic effects. This means there is no actual transit.

% For the two stars in the data set most similar to the sun, delta Pavonis and beta Hydri, a search for exoplanets was conducted using TLS. However, no signals were found.


% \subsection{Results}
% The bRing User Reduction Pipeline (BURP) was created by Remko Stuik to calibrate data of stars observed by the bRing cameras. Using \texttt{burp.detrend}, the data is calibrated and light curves are made. However, BURP does not create proper light curves of stars with $m_V < 4$. A comparison between a light curve created with our calibration and \texttt{burp.detrend} is shown in \ref{burp}. A large amount of scattered data points is still visible in the second plot, particularly below the average magnitude. This means that there are still steps of the data reduction which need to be performed to remove these.
% %
% In \ref{burpzoom} the light curves are displayed on a daily scale. The values of the vertical axes do not correspond, but the relative scale does. The \texttt{burp.detrend} plot displays a wider range of magnitude compared to the first plot. This range could be visible because the removal of trends during the detrending process is not executed properly for these stars. These figures show that the calibration of light curves is possible if the calibration is conducted differently, as described in \ref{sec:methods}. 

% \begin{figure}
%     \centering
%     \includegraphics[width=\textwidth]{figures/burpdetrend.png}
%     \caption{This figure shows two light curves of a star with ASCC-number 2295857. The first plot shows the light curve created using our calibration and the second plot shows the light curve using the calibration of BURP. The time in Julian Date is displayed on the horizontal axis and the apparent magnitude is displayed on the vertical axis.}
    
%     \label{burp}
% \end{figure}
% \begin{figure}
%     \centering
%     \includegraphics[width=\textwidth]{figures/burpdetrend_zoom.png}
%     \caption{This figure shows the difference between our method of calibration and the method of burp. The light curve is shown on a daily scale and both plots have a corresponding vertical scale. The time in Julian Date is displayed on the horizontal axis and the apparent magnitude is displayed on the vertical axis.}
%     \label{burpzoom}
% \end{figure}

%The light curves of the two brightest stars in our data set, Canopus (\ref{Canopus}) and Rigil Kentaurus (\ref{Rigil_Kentaurus}), contain bright peaks of scattered data points. This could be the result of overcorrecting the data affected by clouds, since these peaks occur mostly at very cloudy days.

%In the light curve of Hadar (\ref{Hadar}) the brightness seems to fade at the end of a block of data, directly before the star is not visible for a few months. This dip in brightness is probably caused by a lack of calibration. For several stars, it seems the first and last few days of data should be removed because of this inaccuracy.



%We used the data of the eastern South African camera to compute the periods and uncertainties of six variable stars. The uncertainties were computed by dividing the period by the total number of periods present in the data set. The accuracies of the signals could be improved by combining the signals measured by all cameras.\\
%For beta Doradus (\ref{bet_Dor_fold}), a Delta Cephei, we find a period of 9.84 $\pm$ 0.09 days. This is in accordance with a period 9.8426 days according to \cite{Samus_2004}.\\
%For zeta Phoenicis (\ref{zet_Phe_fold}), an eclipsing binary, we find a period of 1.670 $\pm$ 0.002 days with a SDE of 100.3, which is very high. This is similar to 1.6697671 days as stated in \cite{Samus_2004}.\\
%We find a period of 2.943 $\pm$ 0.008 days for alpha Doradus (\ref{alf_Dor_fold}). \cite{Dubath_2011} shows a period of 2.943 days for this rotating star as well.\\
%The second Delta Cephei, l Carinae (\ref{l_Car_fold}), shows a period of 35.6 $\pm$ 1.2 days. This is close to the 35.53584 days as in \cite{Samus_2004}.\\
%Mu$^1$ Scorpii (\ref{mu_Sco_fold}) is a Beta Lyrae-type eclipsing binary, for which we find a period of 1.446 $\pm$ 0.002 days with a SDE of 41.3. This result is in accordance with \cite{Samus_2004}, where a period of 1.44626907 days is stated.\\
%For the eclipsing binary delta Velorum (\ref{del_Vel_fold}), we find a period of 45.2 $\pm$ 2.0 days with a SDE of 3.8. This agrees with the period of 45.15 days as reported by \cite{Pribulla_2011}. The peak does not cross the threshold value of SDE $>$ 6, however. This could be because TLS was meant to be used for transits, and the eclipses have a different shape. This is also our longest period, making it harder to detect. \\
%The period folds of both l Carinae (\ref{l_Car_fold}) and delta Velorum (\ref{del_Vel_fold}) show that longer periods can be detected in the data measured by the bRing instrument. This was not possible for stars brighter than magnitude 4 before bRing was introduced, because not enough data existed of these stars. As a consequence, no long periods could be detected without bRing.

\subsection{Future research}

In the future, the data reduction of the bright stars should be perfected.
%
There are still a few errors in the data which should be fixed.
%
For example, some invalid values remain.

For the first time, long-term light curves have been made for stars brighter than magnitude 4.
%
The light curves produced in this research can be used to determine variability in bright stars' emission by searching for periodic signals, which can then be matched up with other observations.
%
%The periodic signals discovered in the bRing data can be studied in more detail using different telescopes.

The shapes of the periods of variable stars brighter than magnitude 4 can now be made visible in a period folded light curve.

\section{Conclusions}

\label{sec:conclusion}

We have produced detrended light curves for 157 stars $(M_V<4)$ that are visible from the Southern bRing stations ($\delta \leq -30^o$).
%
These light curves cover over seven years of high cadence photometry, enabling frequency domain studies over seven decades of timescales.
%
The light curves are suitable for pulsational studies and several time domain cases.
%
All data and data reduction scripts are available online.

\begin{acknowledgements}

This research has used the SIMBAD database, operated at CDS, Strasbourg, France \citep{wenger2000}.
%
This work has used data from the European Space Agency (ESA) mission {\it Gaia} (\url{https://www.cosmos.esa.int/gaia}), processed by the {\it Gaia} Data Processing and Analysis Consortium (DPAC, \url{https://www.cosmos.esa.int/web/gaia/dpac/consortium}).
%
To achieve the scientific results presented in this article we made use of the \emph{Python} programming language\footnote{Python Software Foundation, \url{https://www.python.org/}}, especially the \emph{SciPy} \citep{virtanen2020}, \emph{NumPy} \citep{numpy}, \emph{Matplotlib} \citep{Matplotlib}, \emph{emcee} \citep{foreman-mackey2013}, and \emph{astropy} \citep{astropy_1,astropy_2} packages.
%
This publication makes use of VOSA, developed under the Spanish Virtual Observatory project supported by the Spanish MINECO through grant AyA2017-84089.
%
This publication makes use of VOSA, developed under the Spanish Virtual Observatory\footnote{\url{https://svo.cab.inta-csic.es}} project funded by MCIN/AEI/10.13039/501100011033/ through grant PID2020-112949GB-I00.
%
VOSA has been partially updated by using funding from the European Union's Horizon 2020 Research and Innovation Programme, under Grant Agreement 776403 (EXOPLANETS-A). 
%
\end{acknowledgements}

\bibliographystyle{aa}
\bibliography{bib}

\begin{appendix}
In the table below, the following information is given for all 157 stars, from left to right: ASCC- number, HD identifier, V-magnitude according to the bRing catalogue, and the period which was used for detrending in days. The stars are ordered by ASCC-number.
\newpage
\begin{table*}[h]
\centering
\begin{tabular}{|l|l|l|l|HHHHl|}
\hline
\textbf{ASCC}    & \textbf{HD}    & \textbf{Names} & \boldmath$m_V$  & \textbf{AUW}    & \textbf{AUE}    & \textbf{SAW}    & \textbf{SAE}   & \textbf{Period}     \\ \hline
\hline
1721367 & 27376 & upsilon$^4$ Eridani & 3.546 & 42029 & 71155 & 33430 & 77722 & - \\ \hline 
1722082 & 29291 & \makecell[l]{Theemin,\\ upsilon$^2$ Eridani}  & 3.805 & 19570 & 1873 & 15334 & 25559 & - \\ \hline 
1728196 & 44402 & \makecell[l]{Furud,\\ zeta Canis Majoris}  & 3.009 & 26425 & 4937 & 18505 & 25486 & - \\ \hline 
1728326 & 44762 & delta Columbae & 3.841 & 47741 & 42160 & 30414 & 57096 & - \\ \hline 
1730728 & 50013 & kappa Canis Majoris & 3.516 & 45017 & 37468 & 28910 & 54554 & - \\ \hline 
1748375 & 74575 & alpha Pyxidis & 3.68 & 43115 & 41415 & 41515 & 51432 & - \\ \hline 
1762319 & 100407 & xi Hydrae & 3.535 & 35249 & 18853 & 43602 & 43095 & - \\ \hline 
1776717 & 141556 & chi Lupi & 3.956 & 44910 & 70143 & 74660 & 67749 & - \\ \hline 
1781002 & 151680 & \makecell[l]{Larawag,\\ epsilon Scorpii} & 2.259 & 34210 & 45114 & 59976 & 54262 & - \\ \hline 
1790528 & 165135 & \makecell[l]{Alnasl,\\ gamma$^2$ Sagittarii} & 3.239 & 22639 & 11112 & 38527 & 32720 & - \\ \hline 
1821588 & 36597 & epsilon Columbae & 3.862 & 63601 & 95234 & 48186 & 91536 & - \\ \hline 
1822707 & 39425 & \makecell[l]{Wazn,\\ beta Columbae} & 3.105 & 70354 & 99644 & 54385 & 103205 & - \\ \hline 
1831780 & 63032 & c Puppis & 3.621 & 95254 & 120484 & 172314 & 137767 & - \\ \hline 
1839751 & 74006 & beta Pyxidis & 3.96 & 67463 & 81382 & 74510 & 92615 & - \\ \hline 
1862167 & 115892 & iota Centauri & 2.743 & 73709 & 106510 & 141952 & 113897 & - \\ \hline 
1862828 & 117440 & d Centauri & 3.232 & 82682 & 124294 & 166321 & 142042 & - \\ \hline 
1865291 & 123139 & \makecell[l]{Menkent,\\ theta Centauri} & 2.058 & 71728 & 109034 & 114756 & 108006 & - \\ \hline 
1867878 & 129116 & beta Centauri & 3.99 & 84913 & 132249 & 219105 & 146755 & - \\ \hline 
1871021 & 136422 & phi$^1$ Lupi & 3.568 & 67560 & 111915 & 135354 & 110184 & - \\ \hline 
1873533 & 143118 & eta Lupi & 3.414 & 87890 & 150376 & 237593 & 155971 & - \\ \hline 
1875903 & 151890 & \makecell[l]{Xamidimura,\\ mu$^1$ Scorpii} & 2.993 & 80328 & 151181 & 207641 & 144973 & 1.44626907 \\ \hline 
1875934 & 151985 & \makecell[l]{Pipirima,\\ mu$^2$ Scorpii} & 3.548 & 80034 & 150489 & 204726 & 144387 & - \\ \hline 
1880579 & 158408 & \makecell[l]{Lesath,\\ upsilon Scorpii} & 2.674 & 69112 & 134756 & 149201 & 122917 & - \\ \hline 
1880898 & 158926 & \makecell[l]{Shaula,\\ lambda Scorpii} & 1.623 & 58209 & 114228 & 81739 & 129014 & - \\ \hline 
1881897 & 160578 & kappa Scorpii & 2.381 & 86734 & 190031 & 182930 & 168445 & - \\ \hline 
1882798 & 161892 & \makecell[l]{Fuyue,\\ G Scorpii} & 3.183 & 64900 & 138398 & 147147 & 129589 & - \\ \hline 
1887460 & 167618 & eta Sagittarii & 3.131 & 60904 & 133603 & 123343 & 122620 & - \\ \hline 
1901232 & 207971 & \makecell[l]{Aldhanab,\\ gamma Gruis} & 3.003 & 58252 & 128824 & 94915 & 126457 & - \\ \hline 
1905964 & 2262 & kappa Phoenicis & 3.871 & 130543 & 275229 & 136936 & 334749 & - \\ \hline 
1905967 & 2261 & \makecell[l]{Ankaa,\\ alpha Phoenicis} & 2.391 & 125364 & 261656 & 126294 & 337648 & - \\ \hline 
1907532 & 9053 & gamma Phoenicis & 3.44 & 134345 & 278078 & 119534 & 377143 & - \\ \hline 
1909803 & 18622 & theta$^1$ Eridani & 3.211 & 57635 & 126153 & 55993 & 138132 & - \\ \hline 
1911993 & 26967 & alpha Horologii & 3.854 & 151370 & 255602 & 115967 & 360956 & - \\ \hline 
1923547 & 59717 & \makecell[l]{Hadir,\\ sigma Puppis} & 3.26 & 171044 & 231000 & 170738 & 334356 & - \\ \hline 
1925996 & 64440 & a Puppis & 3.709 & 163970 & 218126 & 170931 & 210093 & - \\ \hline 
\end{tabular}
\end{table*}

\newpage

\begin{table*}[h]
\centering
\begin{tabular}{|l|l|l|l|HHHHl|}
\hline
\textbf{ASCC}    & \textbf{HD}    & \textbf{Names} & \boldmath$m_V$  & \textbf{AUW}    & \textbf{AUE}    & \textbf{SAW}    & \textbf{SAE}   & \textbf{Period}     \\ \hline
\hline
1927667 & 66811 & \makecell[l]{Naos,\\ zeta Puppis} & 2.217 & 115762 & 217775 & 123059 & 186477 & - \\ \hline 
1936013 & 78647 & \makecell[l]{Suhail,\\ lambda Velorum} & 2.21 & 134666 & 193189 & 162463 & 334359 & - \\ \hline 
1938697 & 82434 & psi Velorum & 3.582 & 158426 & 201176 & 217702 & 201720 & - \\ \hline 
1943462 & 88955 & q Velorum & 3.832 & 173251 & 199234 & 227958 & 305305 & - \\ \hline 
1959119 & 120307 & nu Centauri & 3.397 & 167311 & 208245 & 262723 & 291221 & - \\ \hline 
1959126 & 120324 & mu Centauri & 3.46 & 171644 & 212955 & 267756 & 339768 & - \\ \hline 
1959843 & 121743 & phi Centauri & 3.811 & 167053 & 214406 & 262361 & 340358 & - \\ \hline 
1959875 & 121790 & upsilon$^1$ Centauri & 3.848 & 181713 & 229943 & 277109 & 365521 & - \\ \hline 
1962822 & 127972 & eta Centauri & 2.328 & 145080 & 194372 & 183128 & 339904 & - \\ \hline 
1964703 & 132058 & beta Lupi & 2.668 & 164615 & 206248 & 221732 & 354055 & - \\ \hline 
1967127 & 136298 & delta Lupi & 3.206 & 152168 & 205969 & 244678 & 235667 & - \\ \hline 
1967246 & 136504 & epsilon Lupi & 3.369 & 176363 & 240548 & 274742 & 377785 & - \\ \hline 
1968585 & 138690 & gamma Lupi & 2.812 & 153114 & 217706 & 231582 & 261977 & - \\ \hline 
1974877 & 152334 & zeta$^1$ Scorpii & 3.608 & 153031 & 246511 & 226171 & 319588 & - \\ \hline 
1976808 & 155203 & eta Scorpii & 3.315 & 153548 & 264762 & 241674 & 393034 & - \\ \hline 
1980004 & 159532 & \makecell[l]{Sargas,\\ theta Scorpii} & 1.851 & 96750 & 157836 & 127172 & 383352 & - \\ \hline 
1981329 & 161471 & iota$^1$ Scorpii & 3.005 & 111101 & 242913 & 219947 & 224807 & - \\ \hline 
1992514 & 181454 & \makecell[l]{Arkab Prior,\\ beta$^1$ Sagittarii} & 3.953 & 149254 & 275843 & 224579 & 357822 & - \\ \hline 
1992594 & 181869 & \makecell[l]{Rukbat,\\ alpha Sagittarii} & 3.95 & 122340 & 243233 & 194315 & 242710 & - \\ \hline 
2000576 & 213009 & delta$^1$ Gruis & 3.959 & 133316 & 271688 & 168182 & 356114 & - \\ \hline 
2003304 & 496 & epsilon Phoenicis & 3.875 & 129175 & 286752 & 135224 & 347961 & - \\ \hline 
2004703 & 6595 & beta Phoenicis & 3.315 & 148123 & 303550 & 147342 & 418886 & - \\ \hline 
2005323 & 9362 & delta Phoenicis & 3.937 & 149132 & 309775 & 155544 & 434821 & - \\ \hline 
2021081 & 63922 & P Puppis & 3.941 & 191324 & 241883 & 204760 & 375992 & - \\ \hline 
2023562 & 68273 & gamma$^2$ Velorum & 1.812 & 91755 & 180554 & 88358 & 407483 & - \\ \hline 
2027459 & 74180 & b Velorum & 3.808 & 200293 & 246277 & 223264 & 379316 & - \\ \hline 
2028223 & 75063 & a Velorum & 3.89 & 200295 & 236193 & 243967 & 371324 & - \\ \hline 
2030506 & 78004 & c Velorum & 3.743 & 200714 & 253268 & 241167 & 394426 & - \\ \hline 
2040152 & 92139 & p Velorum & 3.832 & 211973 & 248356 & 277991 & 395668 & - \\ \hline 
2041127 & 93497 & mu Velorum & 2.709 & 191569 & 242702 & 259890 & 400629 & - \\ \hline 
2050494 & 109787 & tau Centauri & 3.845 & 199681 & 246063 & 292914 & 403599 & - \\ \hline 
2050813 & 110304 & gamma Centauri & 2.421 & 108758 & 169885 & 129350 & 378054 & - \\ \hline 
2056697 & 121263 & zeta Centauri & 2.523 & 155149 & 212851 & 217055 & 380046 & - \\ \hline 
2058847 & 125238 & iota Lupi & 3.542 & 188185 & 246938 & 299346 & 399890 & - \\ \hline 
2061037 & 129056 & alpha Lupi & 2.282 & 122046 & 188434 & 173599 & 400435 & - \\ \hline 
2064147 & 134481 & kappa Lupi & 3.848 & 190900 & 269035 & 311192 & 454330 & - \\ \hline 
2079242 & 158427 & alpha Arae & 2.835 & 181564 & 299495 & 283008 & 497554 & - \\ \hline 
2086192 & 169467 & alpha Telescopii & 3.482 & 166337 & 293712 & 264261 & 452607 & - \\ \hline 
2094623 & 196171 & alpha Indi & 3.104 & 142117 & 279285 & 209775 & 420754 & - \\ \hline 
2098110 & 209952 & \makecell[l]{Alnair,\\ alpha Gruis} & 1.763 & 75698 & 170034 & 101517 & 437843 & - \\ \hline 
2099216 & 214952 & \makecell[l]{Tiaki,\\ beta Gruis} & 2.114 & 97692 & 234739 & 97476 & 421217 & - \\ \hline 
2100006 & 218670 & iota Gruis & 3.878 & 136272 & 279562 & 165335 & 400079 & - \\ \hline 
\end{tabular}
\end{table*}

\newpage

\begin{table*}[h]
\centering
\begin{tabular}{|l|l|l|l|HHHHl|}
\hline
\textbf{ASCC}    & \textbf{HD}    & \textbf{Names}                  & \boldmath$m_V$  & \textbf{AUW}    & \textbf{AUE}    & \textbf{SAW}    & \textbf{SAE}   & \textbf{Period}     \\ \hline
\hline
2104402 & 14228 & phi Eridani & 3.546 & 152455 & 317682 & 148180 & 444632 & - \\ \hline 
2110269 & 39060 & beta Pictoris & 3.851 & 175652 & 272185 & 172186 & 420967 & - \\ \hline 
2111805 & 45348 & \makecell[l]{Canopus,\\ alpha Carinae} & -0.64 & 29 & 545 & 1571 & 60898 & - \\ \hline 
2113099 & 50310 & \makecell[l]{Altaleban,\\ tau Puppis} & 2.93 & 185365 & 284208 & 195685 & 428269 & - \\ \hline 
2117254 & 65575 & chi Carinae & 3.448 & 216554 & 279097 & 236114 & 434479 & - \\ \hline 
2119926 & 74195 & omicron Velorum & 3.588 & 216834 & 286216 & 199237 & 437295 & - \\ \hline 
2120290 & 74956 & delta Velorum & 1.968 & 82402 & 200637 & 129889 & 439146 & 45.15 \\ \hline 
2129944 & 86440 & phi Velorum & 3.533 & 227038 & 276556 & 277796 & 445705 & - \\ \hline 
2141153 & 98718 & pi Centauri & 3.884 & 230878 & 253979 & 308891 & 424507 & - \\ \hline 
2146573 & 105435 & delta Centauri & 2.558 & 167581 & 220369 & 216269 & 385182 & - \\ \hline 
2146914 & 105937 & rho Centauri & 3.954 & 222601 & 249041 & 309345 & 416426 & - \\ \hline 
2148632 & 108483 & sigma Centauri & 3.903 & 211649 & 252105 & 299710 & 411859 & - \\ \hline 
2154691 & 118716 & epsilon Centauri & 2.272 & 119054 & 186701 & 166476 & 406334 & - \\ \hline 
2162355 & 134505 & zeta Lupi & 3.398 & 214278 & 298075 & 351660 & 474895 & - \\ \hline 
2182568 & 165024 & theta Arae & 3.664 & 183731 & 312778 & 302575 & 496629 & - \\ \hline 
2195224 & 215789 & epsilon Gruis & 3.48 & 146373 & 328673 & 194265 & 479643 & - \\ \hline 
2198437 & 6882 & \makecell[l]{Wurren,\\ zeta Phoenicis} & 3.977 & 157643 & 329094 & 166608 & 470090 & 1.6697671 \\ \hline 
2199019 & 10144 & \makecell[l]{Achernar,\\ alpha Eridani} & 0.504 & 2000 & 34653 & 61763 & 299143 & - \\ \hline 
2203013 & 29305 & alpha Doradus & 3.232 & 178256 & 309163 & 163391 & 447895 & 2.943 \\ \hline 
2212858 & 71129 & epsilon Carinae A & 2.009 & 66658 & 186685 & 122020 & 446786 & - \\ \hline 
2217847 & 80404 & \makecell[l]{Aspidiske,\\ iota Carinae} & 2.236 & 106861 & 235433 & 184823 & 452737 & - \\ \hline 
2218411 & 81188 & kappa Velorum & 2.469 & 138156 & 239710 & 185936 & 423057 & - \\ \hline 
2219507 & 82668 & N Velorum & 3.165 & 233276 & 275917 & 279514 & 436596 & - \\ \hline 
2229331 & 90853 & s Carinae & 3.818 & 239051 & 303514 & 305420 & 463960 & - \\ \hline 
2234363 & 94510 & u Carinea & 3.776 & 240649 & 285660 & 319970 & 447691 & - \\ \hline 
2237409 & 96918 & x Carinea & 3.911 & 242168 & 292536 & 337653 & 458901 & - \\ \hline 
2246690 & 106490 & \makecell[l]{Imai,\\ delta Crucis} & 2.779 & 192208 & 232750 & 250937 & 377873 & - \\ \hline 
2248482 & 108903 & \makecell[l]{Gacrux,\\ gamma Crucis} & 1.656 & 42955 & 89368 & 66015 & 382630 & - \\ \hline 
2250231 & 111123 & \makecell[l]{Mimosa,\\ beta Crucis} & 1.294 & 19909 & 38198 & 100852 & 414253 & - \\ \hline 
2268985 & 143474 & iota$^1$ Normae & 3.641 & 220918 & 312421 & 360193 & 524270 & - \\ \hline 
2274286 & 151249 & eta Arae & 3.766 & 208546 & 325429 & 326456 & 531368 & - \\ \hline 
2275069 & 152786 & zeta Arae & 3.108 & 210763 & 319779 & 340171 & 517338 & - \\ \hline 
2277091 & 157244 & beta Arae & 2.819 & 203779 & 318868 & 291356 & 513729 & - \\ \hline 
2277100 & 157246 & gamma Arae & 3.315 & 206553 & 324187 & 308237 & 444850 & - \\ \hline 
2286020 & 193924 & \makecell[l]{Peacock,\\ alpha Pavonis} & 1.917 & 64754 & 160068 & 121707 & 447170 & - \\ \hline 
2286969 & 198700 & beta Indi & 3.645 & 186150 & 371738 & 254237 & 533966 & - \\ \hline 
2290853 & 219571 & gamma Tucanae & 3.992 & 163684 & 372844 & 207675 & 488036 & - \\ \hline 
2293937 & 12311 & alpha Hydri & 2.851 & 161910 & 355190 & 185208 & 499254 & - \\ \hline 
2295857 & 23817 & beta Reticuli & 3.832 & 193930 & 373286 & 197764 & 536622 & - \\ \hline 
2296439 & 27256 & alpha  Reticuli & 3.331 & 192775 & 342198 & 185998 & 488330 & - \\ \hline 
\end{tabular}
\end{table*}

\newpage

\begin{table*}[h]
\centering
\begin{tabular}{|l|l|l|l|HHHHl|}
\hline
\textbf{ASCC}    & \textbf{HD}    & \textbf{Names}                  & \boldmath$m_V$  & \textbf{AUW}    & \textbf{AUE}    & \textbf{SAW}    & \textbf{SAE}   & \textbf{Period}     \\ \hline
\hline
2298409 & 37350 & beta Doradus & 3.803 & 196565 & 325820 & 205225 & 488500 & 9.8426 \\ \hline 
2300928 & 50241 & alpha Pictoris & 3.243 & 207984 & 319582 & 244671 & 469327 & - \\ \hline 
2307136 & 76728 & c Carinae & 3.831 & 238611 & 288090 & 317961 & 452865 & - \\ \hline 
2308342 & 79447 & i Carinae & 3.953 & 239956 & 295042 & 315219 & 472620 & - \\ \hline 
2311128 & 84810 & l Carinae & 3.74 & 242772 & 311485 & 337630 & 513640 & 35.53584 \\ \hline 
2315295 & 89388 & q Carinae & 3.369 & 245610 & 295683 & 305155 & 468284 & - \\ \hline 
2317331 & 91465 & p Carinae & 3.359 & 236697 & 297669 & 310154 & 474416 & - \\ \hline 
2318884 & 93030 & theta Carinae & 2.73 & 194361 & 262524 & 246325 & 473431 & - \\ \hline 
2325848 & 100841 & lambda Centauri & 3.114 & 240881 & 297572 & 329583 & 481397 & - \\ \hline 
2333058 & 107446 & \makecell[l]{Ginan, \\epsilon Crucis} & 3.591 & 240707 & 277907 & 364290 & 434380 & - \\ \hline 
2333718 & 108248 & \makecell[l]{Acrux,\\ alpha$^1$ Crucis} & 1.039 & 675 & 12055 & 80989 & 315550 & - \\ \hline 
2345078 & 122451 & \makecell[l]{Hadar,\\ beta Centauri} & 0.58 & 390 & 11996 & 79362 & 312783 & - \\ \hline 
2348878 & 128620 & \makecell[l]{Rigil Kentaurus,\\ alpha Centauri A} & -0.057 & 50 & 5243 & 32508 & 81824 & - \\ \hline 
2349085 & 128898 & alpha Circini & 3.174 & 235128 & 313339 & 371518 & 588465 & - \\ \hline 
2354207 & 141891 & beta Trianguli Australis & 2.826 & 219234 & 308363 & 351282 & 599093 & - \\ \hline 
2356113 & 145544 & delta Trianguli Australis & 3.846 & 229778 & 317196 & 368887 & 558970 & - \\ \hline 
2361416 & 158094 & delta Arae & 3.589 & 212151 & 337792 & 350269 & 567363 & - \\ \hline 
2362160 & 160635 & eta Pavonis & 3.585 & 214507 & 351517 & 350263 & 604918 & - \\ \hline 
2370345 & 211416 & alpha Tucanae & 2.855 & 166787 & 369612 & 195406 & 505740 & - \\ \hline 
2381455 & 57623 & delta Volantis & 3.964 & 253969 & 354612 & 278629 & 546272 & - \\ \hline 
2383925 & 71878 & beta Volantis & 3.763 & 243511 & 343635 & 312020 & 529055 & - \\ \hline 
2385516 & 78045 & alpha Volantis & 3.994 & 246484 & 325837 & 335319 & 533175 & - \\ \hline 
2386073 & 80007 & \makecell[l]{Miaplacidus, \\ beta Carinae}  & 1.661 & 41869 & 157056 & 193001 & 562705 & - \\ \hline 
2387890 & 85123 & upsilon Carinae A & 2.993 & 231904 & 317996 & 335441 & 533850 & - \\ \hline 
2395368 & 102249 & lambda Muscae & 3.624 & 246222 & 314548 & 362914 & 505857 & - \\ \hline 
2399317 & 109668 & alpha Muscae & 2.681 & 210319 & 294455 & 329596 & 501058 & - \\ \hline 
2399980 & 110879 & bet Muscae & 3.286 & 250085 & 314634 & 360350 & 552228 & - \\ \hline 
2410740 & 135382 & gamma Trianguli Australis & 2.875 & 236493 & 318439 & 396263 & 593142 & - \\ \hline 
2417375 & 150798 & \makecell[l]{Atria, \\ alpha Trianguli Australis} & 1.911 & 99702 & 154691 & 185686 & 542791 & - \\ \hline 
2424942 & 190248 & delta Pavonis & 3.529 & 196507 & 373301 & 282071 & 629418 & - \\ \hline 
2425895 & 197051 & bet Pavonis & 3.415 & 186129 & 379733 & 271656 & 617554 & - \\ \hline 
2430714 & 2151 & bet Hydri & 2.792 & 231346 & 457263 & 215032 & 640299 & - \\ \hline 
2438259 & 24512 & gamma Hydri & 3.263 & 234194 & 427581 & 194968 & 624806 & - \\ \hline 
2445952 & 55865 & gamma$^2$ Volantis & 3.756 & 265796 & 376123 & 273270 & 554573 & - \\ \hline 
2455523 & 89080 & omega Carinae & 3.293 & 258150 & 368822 & 343038 & 553292 & - \\ \hline 
2456263 & 90589 & I Carinae & 3.983 & 290455 & 362275 & 371888 & 587257 & - \\ \hline 
2472748 & 129078 & alpha Apodis & 3.813 & 284354 & 342175 & 402492 & 563655 & - \\ \hline 
2481765 & 147675 & gamma Apodis & 3.861 & 285336 & 378340 & 381377 & 658956 & - \\ \hline 
2493279 & 188228 & epsilon Pavonis & 3.946 & 225994 & 434979 & 289883 & 667901 & - \\ \hline 
2497346 & 205478 & nu Octantis & 3.73 & 218645 & 434742 & 237142 & 625119 & - \\ \hline 
\end{tabular}
\end{table*}
\end{appendix}

\end{document}
